\documentclass{jcgt}

\setciteauthor{Matthias Raab and Laurent Belcour and Frankie Liu and Jamie Portsmouth}
\setcitetitle{The Minimal Retroreflective Microfacet Model (MRMM)}
%\setheadtitle{Abbreviated title, only if full title won't fit in page headers}

% Mark submissions with the date of submission using the following line:
\submitted{\today}

% Once an article is accepted accepted, switch to the following line and comment the preceding one. The editor will supply the argument values.
\accepted{yyyy-mm-dd}{yyyy-mm-dd}{yyyy-mm-dd}{Editor Name}{volume}{issue}{1}{6}{yyyy}
\seturl{http://jcgt.org/published/vol/issue/num/}

\usepackage{amsmath}

\newcommand{\vecv}{\mathbf{v}}
\newcommand{\vecvv}{\mathbf{v'}}
\newcommand{\vecl}{\mathbf{l}}
\newcommand{\vecll}{\mathbf{l'}}
\newcommand{\vecn}{\mathbf{n}}
\newcommand{\vech}{\mathbf{h}}
\newcommand{\vecb}{\mathbf{b}}

\newcommand{\vdot}[2]{\left\langle #1, #2 \right\rangle}


\begin{document}

\author{Matthias Raab\\NVIDIA
        \and Laurent Belcour\\Intel
        \and Frankie Liu\\NVIDIA
        \and Jamie Portsmouth\\Autodesk
       }

\title{The Minimal Retroreflective Microfacet Model}

\maketitle
\thispagestyle{firstpagestyle}

\begin{abstract}
  \small
  We introduce the ``Minimal Retroreflective Microfacet'' (MRM) model, which provides a simple scheme for
  modifying a pre-existing microfacet BRDF implementation to produce a visual plausible retroreflective result.
  \end{abstract}

\section{Introduction}




\section{Microfacet BRDF Models}

\begin{figure}
  \begin{center}
    \begin{tabular} {|l|l|}
      \hline
      $\vecv$ & normalized vector to viewer \\
      \hline
      $\vecl$ & normalized vector to light \\
      \hline
      $\vecn$ & surface normal \\
      \hline
      $\mathbf{t_1}, \mathbf{t_2}$ & tangent and bitangent  \\
      \hline
      $\vech$ & half vector, $\vech = \vech(\vecv, \vecl) = \frac{\vecv + \vecl}{\lVert \vecv + \vecl \rVert}$ \\
      \hline
      $\vdot{\mathbf{x}}{\mathbf{y}}$ & dot product of $\mathbf{x}$ and $\mathbf{y}$ \\
      \hline
      $\text{reflect}(\mathbf{x}, \mathbf{m})$ & reflection of vector $\mathbf{x}$ on surface with normal $\mathbf{m}$, i.e. $-\mathbf{x} + 2 \vdot{\mathbf{x}}{\mathbf{m}} \mathbf{m}$ \\
      \hline
      $\Omega$ & hemisphere of directions around the normal \\
      \hline
    \end{tabular}
  \end{center}
  \caption{Notations used.}
  \label{fig:notation}
\end{figure}

Using the notations from figure \ref{fig:notation}, a microfacet BRDF \cite{WalterMicrofacet, HeitzMicrofacet} takes the form
\begin{equation}\label{eq:microfacet}
  f(\vecv, \vecl) = \frac{D(\vech) G_{2}(\vecv,\vecl)}{4 \vdot{\vecv}{\vecn} \vdot{\vecl}{\vecn}}
\end{equation}
where $D$ is the distribution of microfacets, normalized to fulfill $\int_\Omega D(\vech) \vdot{\vecn}{\vech} d\vech = 1$,
and $G_{2}$ is the shadowing-masking function.
The shadowing-masking function combines the visibility masking function $G_1(\vecv, \vecl)$ for view direction and
$G_1(\vecl, \vecv)$ for light direction:
\begin{equation}
G_{2}(\vecv, \vecl) = G_{2}\left(G_1(\vecv, \vecl), G_1(\vecl, \vecv) \right)
\end{equation}
Mathematically, the BRDF is plausible if $G_{2}$ is symmetric w.r.t. to exchanging $\vecv$ and $\vecl$, fulfills
$G_{2}(\vecv, \vecl) \leq G_1(\vecv, \vecl)$, and if $G_1$ further establishes the correct projected area of visible micro-surfaces \cite{HeitzMicrofacet}
\begin{equation}\label{eq:visible}
  \vdot{\vecv}{\vecn} = \int_{\lbrace\vech \in \Omega: \vdot{\vecv}{\vech} \geq 0\rbrace} D(\vech) G_1(\vecv, \text{reflect}(\vecv, \vech)) \vdot{\vecv}{\vech} d\vech.
\end{equation}
Choices for $G_1$ with this property are the Smith masking function
\begin{equation}
  G_1(\vecv, \vecl) = G_{1,\text{Smith}}(\vecv) = \frac{\vdot{\vecv}{\vecn}}{\int_{\lbrace\vech \in \Omega: \vdot{\vecv}{\vech} \geq 0\rbrace} D(\vech) \vdot{\vecv}{\vech} d\vech},
\end{equation}
where analytic expressions are available for Beckmann and GGX distributions, and the v-cavity masking function
\begin{equation}
  G_1(\vecv, \vecl) = G_{1,\text{vc}}(\vdot{\vecv}{\vecn},\vdot{\vecn}{\vech},\vdot{\vecv}{\vech}) =
  \min\left\lbrace\frac{2\vdot{\vecv}{\vecn}\vdot{\vecn}{\vech}}{\vdot{\vecv}{\vech}}, 1\right\rbrace,
\end{equation}
which is generally applicable for distributions with symmetry around the normal.

\section{Back-vector Modification}

If instead of half-vector $\vech$ the back-vector $\vecb$, defined as
\begin{equation}
  \vecb(\vecv, \vecl) = \vech(\vecvv, \vecl) = \frac{\vecvv + \vecl}{\lVert \vecvv + \vecl \rVert}, \text{ with } \vecvv = \text{reflect}(\vecv, \vecn),
\end{equation}
is used in equation \eqref{eq:microfacet}, the resulting BRDF is retro-reflective \cite{BelcourBackvector}.
Note that instead of explicitly defining a BRDF using the back-vector, we can simply use $\vecvv$ instead of $\vecv$, i.e. we define $f_{\text{retro}}(\vecv, \vecl) := f(\vecvv, \vecl)$ such that the half-vector becomes the back-vector.
In the following we show that this BRDF is plausible, i.e. symmetry and energy conservation are guaranteed.

\section{Plausibility of the Back-vector Modification}

In analogy to $\vecvv$ we let $\vecll := \text{reflect}(\vecl, \vecn)$.
First we observe
\begin{equation}\label{eq:nk}
  \begin{split}
    \vdot{\vecvv}{\vecn} &= \vdot{\vecv}{\vecn} \text{ and}\\
    \vdot{\vecl}{\vecn} &= \vdot{\vecll}{\vecn}.
  \end{split}
\end{equation}
For the back-vector we show $\vech(\vecvv, \vecl) = \text{reflect}\left(\vech(\vecll, \vecv), \vecn\right)$, as
\begin{equation}
  \begin{split}
  \vecvv + \vecl &=
  \left(-\vecv + 2 \vdot{\vecv}{\vecn} \vecn \right) + \vecl \\
  &= \left(-\vecv + 2 \vdot{\vecv}{\vecn} \vecn \right) + \left(- \vecll + 2 \vdot{\vecll}{\vecn} \vecn \right)\\
  &= -(\vecll + \vecv) + 2 \vdot{(\vecll + \vecv)}{\vecn} \vecn =
  \text{reflect}(\vecll + \vecv, \vecn)
  \end{split}
\end{equation}
and therefore $\lVert \vecvv + \vecl \rVert = \lVert \vecll + \vecv \rVert$ as well.
From that we can conclude
\begin{equation}\label{eq:nh}
  \begin{split}
    \vdot{\vecn}{\vech(\vecvv, \vecl)} &=
    \vdot{\vecn}{\vech(\vecll, \vecv)}\\
    \vdot{\mathbf{t_1}}{\vech(\vecvv, \vecl)} &=
    -\vdot{\mathbf{t_1}}{\vech(\vecll, \vecv)}\\
    \vdot{\mathbf{t_2}}{\vech(\vecvv, \vecl)} &=
    -\vdot{\mathbf{t_2}}{\vech(\vecll, \vecv)}.\\
    \end{split}
\end{equation}
As anisotropic Phong, Beckmann, and GGX distributions are reflection-symmetric around the normal, i.e. they take the form $D(\vech) = D\left(\vdot{\vecn}{\vech}, \lvert \vdot{\mathbf{t_1}}{\vech} \rvert, \lvert \vdot{\mathbf{t_2}}{\vech} \rvert\right)$, we can thus infer that using them with the back-vector obeys symmetry.
Further, the reflection symmetry of $D$ along with equation \eqref{eq:nk} gives
\begin{equation}
  \begin{split}
  G_{1,\text{Smith}}(\vecv) &=
  \frac{\vdot{\vecv}{\vecn}}{\int_{\lbrace\vech \in \Omega: \vdot{\vecv}{\vech} \geq 0\rbrace} D(\vech) \vdot{\vecv}{\vech} d\vech} \\
  &= \frac{\vdot{\vecvv}{\vecn}}{\int_{\lbrace\vech \in \Omega: \vdot{\vecvv}{\vech} \geq 0\rbrace} D(\vech) \vdot{\vecvv}{\vech} d\vech} =
  G_{1,\text{Smith}}(\vecvv)
  \end{split}
\end{equation}
for Smith-masking. Next we show
\begin{equation}
  \vdot{\vecvv}{\vecl} = -\vdot{\vecv}{\vecl} + 2 \vdot{\vecv}{\vecn}\vdot{\vecl}{\vecn}
  = \vdot{\vecv}{-\vecl + 2 \vdot{\vecl}{\vecn}\vecn} = \vdot{\vecv}{\vecll}
\end{equation}
and therefore
\begin{equation}\label{eq:kh}
  \vdot{\vecvv}{\vech(\vecvv, \vecl)} =
  \frac{1 + \vdot{\vecl}{\vecvv}}{\lVert \vecvv + \vecl \rVert} =
  \frac{1 + \vdot{\vecll}{\vecv}}{\lVert \vecll + \vecv \rVert} =
  \vdot{\vecll}{\vech(\vecll,\vecv)}.
\end{equation}
Using this along with equations \eqref{eq:nh} and \eqref{eq:nk} we get
\begin{equation}
  \begin{split}
  G_1(\vecvv, \vecl) &=
  G_{1, \text{vc}}\left( \vdot{\vecvv}{\vecn}, \vdot{\vecn}{\vech(\vecvv, \vecl)}, \vdot{\vecvv}{\vech(\vecvv, \vecl)} \right)\\
  &=  G_{1, \text{vc}}\left( \vdot{\vecv}{\vecn}, \vdot{\vecn}{\vech(\vecll, \vecv)}, \vdot{\vecll}{\vech(\vecll, \vecv)} \right) \\
  &=  G_{1, \text{vc}}\left( \vdot{\vecv}{\vecn}, \vdot{\vecn}{\vech(\vecv, \vecll)}, \vdot{\vecll}{\vech(\vecv, \vecll)} \right) \\
  &= G_1(\vecv, \vecll)
  \end{split}
\end{equation}
and similarly $G_1(\vecl, \vecvv) = G_1(\vecll, \vecv)$ for the v-cavity masking function.
In summary, we can conclude the symmetry of the BRDF
\begin{equation}
  \begin{split}
  f_{\text{retro}}(\vecv, \vecl) =
  f(\vecvv, \vecl) &=
  \frac{D(\vech(\vecvv, \vecl)) G_{2}(G_1(\vecvv, \vecl), G_1(\vecl, \vecvv))}{4 \vdot{\vecvv}{\vecn} \vdot{\vecl}{\vecn} }\\
  &=
  \frac{D(\vech(\vecll, \vecv)) G_{2}(G_1(\vecv, \vecll), G_1(\vecll, \vecv))}{4 \vdot{\vecv}{\vecn} \vdot{\vecll}{\vecn} } =
  f(\vecll, \vecv) =
  f_{\text{retro}}(\vecl, \vecv).
  \end{split}
\end{equation}
For the albedo we simply have
\begin{equation}
  \rho_{\text{retro}}(\vecv) =
  \int_\Omega f_{\text{retro}}(\vecv, \vecl) \vdot{\vecl}{\vecn} d\vecl =
  \int_\Omega f(\vecvv, \vecl) \vdot{\vecl}{\vecn} d\vecl = \rho(\vecvv)
\end{equation}
i.e. we get the albedo of the regular microfacet BRDF for the reflected direction.
Which, using equation \eqref{eq:visible} and $\frac{d\vech(\vecvv, \vecl)}{d\vecl} = \frac{1}{4 \vdot{\vecl}{\vech(\vecvv, \vecl)}}= \frac{1}{4 \vdot{\vecvv}{\vech(\vecvv, \vecl)}}$ \cite{WalterMicrofacet}, can be shown to fulfill energy conservation
\begin{equation}
  \begin{split}
  \rho(\vecvv) &=
  \int_\Omega \frac{D(\vech(\vecvv, \vecl)) G_{2}(\vecvv,\vecl)}{4 \vdot{\vecvv}{\vecn} \vdot{\vecl}{\vecn}}\vdot{\vecl}{\vecn} d\vecl\\
  &\leq
  \int_{\lbrace\vech \in \Omega: \vdot{\vecvv}{\vech} \geq 0\rbrace} \frac{D(\vech) G_1(\vecvv,\text{reflect}(\vecvv, \vech))}{\vdot{\vecvv}{\vecn}}\vdot{\vecvv}{\vech} d\vech
  = 1.
  \end{split}
\end{equation}

\section{Implementation Notes}

Given an implementation of a regular microfacet BRDF, extending it to retro-reflection is straightforward:

\begin{itemize}
\item
  Evaluation merely needs to replace $\vecv$ with $\vecvv$ upfront.
\item
  Similarly, importance sampling of $\vecl$ given $\vecv$ can be realized by replacing $\vecv$ with $\vecvv$ upfront and then importance sampling the regular microfacet BRDF.
  This may include low variance sampling using the domain of visible microfacets \cite{HeitzIS}.
\item
  As the albedos of standard BRDF and retro-reflective BRDF are essentially identical, compensating for energy loss in the sense of \cite{KelemenBRDF} can be realized using the same data tables.
\end{itemize}



\bibliographystyle{jcgt}
\bibliography{citations}

\section*{Author Contact Information}

\hspace{-2mm}\begin{tabular}{p{0.48\textwidth}p{0.48\textwidth}}
Matthias Raab \newline
\href{mailto:matthias.raab@gmail.com}{matthias.raab@gmail.com} \newline
&
Laurent Belcour \newline
\href{mailto:laurent.belcour@gmail.com}{laurent.belcour@gmail.com} \\
Frankie Liu \newline
\href{mailto:frankie.liu@gmail.com}{frankie.liu@gmail.com} \newline
&
Jamie Portsmouth \newline
\href{mailto:jamports@mac.com}{jamports@mac.com} \newline
\end{tabular}


\end{document}


